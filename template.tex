\documentclass[xcolor=table,slidestop,compress,mathserif]{beamer}
\usepackage[no-math]{fontspec}
\usepackage{xeCJK}
\setCJKmainfont[BoldFont={Adobe Heiti Std}]{Adobe Song Std}
\punctstyle{kaiming}
\usepackage{amssymb}
\usepackage{listings}
\setbeamertemplate{items}[circle]
\usetheme{Warsaw}
\defbeamertemplate*{footline}{shadow theme}
{%
  \leavevmode%
  \hbox{\begin{beamercolorbox}[wd=.5\paperwidth,ht=2.5ex,dp=1.125ex,leftskip=.3cm
      plus1fil,rightskip=.3cm]{author in head/foot}%
    \usebeamerfont{author in
      head/foot}\insertshortinstitute\hfill\insertshortauthor
  \end{beamercolorbox}%
  \begin{beamercolorbox}[wd=.5\paperwidth,ht=2.5ex,dp=1.125ex,leftskip=.3cm,rightskip=.3cm
    plus1fil]{title in head/foot}%
    \usebeamerfont{title in
      head/foot}\insertshorttitle\hfill\insertframenumber\,/\,\inserttotalframenumber%
  \end{beamercolorbox}}%
  \vskip0pt%
}
% \usetheme{Montpellier}
% \usetheme{Madrid}
% \usetheme{Singapore}
\usepackage{graphicx}
% added by https://github.com/farseerfc/bm-chapter8/blob/master/chapter8.tex
\lstset{tabsize=4, %
  frame=shadowbox, %把代码用带有阴影的框圈起来
  commentstyle=\color{red!50!green!50!blue!50},%浅灰色的注释
  rulesepcolor=\color{red!20!green!20!blue!20},%代码块边框为淡青色
  keywordstyle=\color{blue!90}\bfseries, %代码关键字的颜色为蓝色,粗体
  showstringspaces=false,%不显示代码字符串中间的空格标记
  stringstyle=\ttfamily, % 代码字符串的特殊格式
  keepspaces=true, %
  breakindent=22pt, %
  numbers=left,%左侧显示行号
  stepnumber=1,%
  numberstyle=\tiny, %行号字体用小号
  basicstyle=\footnotesize, %
  showspaces=false, %
  flexiblecolumns=true, %
  breaklines=true, %对过长的代码自动换行
  breakautoindent=true,%
  breakindent=4em, %
  aboveskip=1em, %代码块边框
  %% added by http://bbs.ctex.org/viewthread.php?tid=53451
  fontadjust,
  captionpos=t,
  framextopmargin=2pt,framexbottommargin=2pt,abovecaptionskip=-3pt,belowcaptionskip=3pt,
  % xleftmargin=4em,xrightmargin=4em, % 设定listing左右的空白
  texcl=true,
  % 设定中文冲突,断行,列模式,数学环境输入,listing数字的样式
  extendedchars=false,columns=flexible,mathescape=true
  numbersep=-1em
}

\pgfdeclareimage[height=0.618cm]{logo}{figures/sjtulogoblue.png}
\logo{\pgfuseimage{logo}}

\def\hilite<#1>{%
  \temporal<#1>{\color{gray}}{\color{blue}}%
  {\color{blue!25}}}
% ------------------------------------------
\title{Beamer Template}
\author{Quanyang Liu}
\institute[CS SJTU]{Department of Computer Science and Engineering \\ Shanghai
  Jiao Tong University}
\date{\today}
\begin{document}
\frame{\titlepage}

\begin{frame}<beamer>[shrink=10]{Outline}
  \tableofcontents[sectionstyle=show,subsectionstyle=hide]
\end{frame}

\AtBeginSubsection[]{
  \begin{frame}<trans|beamer>[shrink=25]{Outline}
    \tableofcontents[sectionstyle=show/shaded,subsectionstyle=show/shaded/hide]
  \end{frame}}
% ------------------------------------------
\section{one}
\subsection{a}
\begin{frame}
  \frametitle{title test}
  \begin{enumerate}
    \pause \item ahh
    \pause \item 2333
    \begin{displaymath}
      \lim_{x \to 0} \frac{ \sin x}{x} = 0
    \end{displaymath}
  \end{enumerate}
\end{frame}
% ------------------------------------------
\subsection{b}
\begin{frame}
  \frametitle{2333}
  \pause
  \begin{equation}
    \sum_{i=1}^n \frac{1}{n!} = \mathrm{e}
  \end{equation}
\end{frame}
% ------------------------------------------
% \section{three}
% \begin{frame}
%   \pgfdeclareimage[width=\textwidth]{checklist}{2014-07-15-183139_1366x768_scrot.png}
%   \begin{figure}
%     \pgfuseimage{checklist}
%   \end{figure}
% \end{frame}
% ------------------------------------------
% \begin{frame}
%   \centering
%   \includegraphics[width=\textwidth]{sjtulogoblue.png}
% \end{frame}
% ------------------------------------------
\section{a}
\subsection{aa}
\begin{frame}
  \setbeamercolor{uppercol}{fg=white,bg=green!50!gray}%
  \setbeamercolor{lowercol}{fg=black,bg=green!20}%
  \begin{columns}[c]
    \begin{column}{0.4\textwidth}
      \begin{beamerboxesrounded}[upper=uppercol,lower=lowercol,shadow=true]{Theorem}
        $A = B$ \\
        $B = C$
      \end{beamerboxesrounded}
    \end{column}
    \pause
    \begin{column}{0.05\textwidth}
      \centering
      $\Rightarrow$
    \end{column}
    \pause
    \begin{column}{0.4\textwidth}
      \begin{beamerboxesrounded}{Theorem}
        $A=C?$
      \end{beamerboxesrounded}
    \end{column}
  \end{columns}
\end{frame}
% ------------------------------------------
\begin{frame}
  \rowcolors[]{1}{blue!20}{blue!10}
  \begin{tabular}{l!{\vrule}c
      <{\onslide<2->}c
      <{\onslide<3->}c
      <{\onslide<4->}c
      <{\onslide}c}
    Class & A & B & C & D \\
    X     & 1 & 2 & 3 & 4 \\
    Y     & 3 & 4 & 5 & 6 \\
    Z     & 5 & 6 & 7 & 8
  \end{tabular}
  \begin{enumerate}
  \item<5->
  \item<6->
  \item<7->
  \end{enumerate}
  \hypertarget{abcd}{}
\end{frame}
% ------------------------------------------
\subsection{bb}
\begin{frame}
  \begin{itemize}
    \hilite<1> \item Everything
    \hilite<2> \item that has
    \hilite<3> \item beginning
    \hilite<4> \item has end.
  \end{itemize}
  \hyperlink{abcd}{\beamergotobutton{abcd}}
\end{frame}
% ------------------------------------------
\begin{frame}[fragile]
  \frametitle{title}
  \begin{columns}
    \begin{column}{0.5\textwidth}
      test
    \end{column}
    \pause
    \begin{column}{0.5\textwidth}
      \begin{lstlisting}[language=C,escapechar=@]
#include <stdio.h>

int main(int argc, char *argv[5])
{
  int i = 0; //ajbk
  scanf("%d", i);
  return 0;
}
    \end{lstlisting}
    \end{column}
  \end{columns}
\end{frame}
% ------------------------------------------
\setbeamertemplate{background canvas}[vertical shading]
[bottom=white,top=structure.fg!25]
\begin{frame}
  \begin{center}
    {\huge \emph{\textcolor[rgb]{0.50,0.00,1.00}{Thank  ~you!
          \\   \vspace{1cm} Any Question?}}}
  \end{center}
  \vspace{1.5cm}\large
  \begin{flushright}
    \begin{tabular}{ll}
      {\sc Author}: & \textsf{Quanyang Liu}\\
      {\sc Email}: & \href{mailto:lqymgt@gmail.com}
      {\color{blue!70}lqymgt@gmail.com}\\
    \end{tabular}
  \end{flushright}
\end{frame}
% ------------------------------------------
\end{document}
